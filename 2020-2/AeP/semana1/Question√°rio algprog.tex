\documentclass[a4paper, 12pt]{article} 
\usepackage[top=3cm, bottom=2cm, left=3cm, right=2cm]{geometry}
\usepackage[utf8]{inputenc}
\usepackage{xspace}

\begin{document}

\begin{center}
\textbf{Resolução do Questionário de algoritmos e programação}

\tiny{Aluna: Anna Gabriele Marques de Oliveira}
\end{center}
Respostas:
\begin{itemize}
\item A principal função da memória CACHE é de otimizar a velocidade de processamento, fazendo com que o processador, não necessite tanto da memória RAM para dar prosseguimento na execução de um determinado programa. A memória CACHE é um armazenamento de menor quantidade de memória, para armazenar dados específicos, como os dados mais importantes e os mais utilizados.

 Assim, a memória CACHE é muito mais rápida do que a memória RAM, por uma série de fatores que permitem que o tempo ocioso do processador seja menor, já que o mesmo possui um alto desempenho comparado as memórias. Fatores como por exemplo, o da memória CACHE ser exclusiva da CPU e por isso possuir uma velocidade de acesso maior.
\vspace{0.3cm} 

\item Dispositivos de entrada tem como função permitirem a interação de um usuário com um computador, eles são os meios pelos quais, o usuário é capaz de inserir informações a serem processadas pela máquina. Dispositivos de entrada podem ser exemplificados por: mouses, teclados e telas touch screen.

Dispositivos de saída tem como função permitirem que o usuário tenha acesso as informações processadas pelo computador, a serem trasmitidas em uma linguagem que o usuário possa utilizar. São esses: Monitores, impressoras e projetores.
\vspace{0.3cm}

\item O sistema operacional é responsável por organizar e interligar todas as partes de um computador, tanto software: onde ele gerencia a execução de todos os programas e capacita a instalação de novos programas, faz a segurança e a comunicação com o usuário; quanto Hardware: onde ele faz a comunicação da CPU com todos os outros dispositivos externos, como as memórias e os sistemas de entrada e saída.

O sistema operacional serve de interface entre computador e o ser humano.

\vspace{0.3cm}
\item Os registradores são as memórias que estão dentro da CPU, eles estão no topo da cadeia de eficiência, e também de custo entre todos os tipos de memória. Eles armazenam n bits por um curto prazo de tempo, onde são armazenados os dados para executar um determinado programa. Esses dados são movidos da mamória RAM para todos os registradores uma vez que o programa será inicializado, e retornam à memória RAM uma vez feita a execução do programa pelo processador.


\end{itemize}

\end{document}