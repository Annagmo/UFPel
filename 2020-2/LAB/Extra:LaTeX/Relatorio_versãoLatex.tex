\documentclass{article}
\usepackage[utf8]{inputenc}
\usepackage{graphicx}
\usepackage[left=3cm,right=2cm,top=3cm,bottom=2cm]{geometry}
\usepackage[brazil]{babel}
\usepackage{pslatex}
\usepackage{xcolor}
\usepackage{enumerate}



\begin{document}

\title{Trabalho de Laboratório da computação}
\author{Anna Gabriele}
\date{\today}
\maketitle

\begin{abstract}
Este trabalho mencionará o básico da formatação em \LaTeX{} que o trabalho de Laboratório da computação requisita, e também outras particularidades da formatação em \LaTeX{} que tornará esse trabalho como uma referência útil no decorrer do curso.
\end{abstract}

%A formatação a seguir é para a criação de um sumário. 
\tableofcontents
\clearpage
%-----------------------

\section{Seção Principal}
Abaixo os tipos de localizações serão mostrados nas seções secundárias.

\subsection{Seção Centralizada}
\begin{center}
Em um texto centralizado cria-se uma linha \textit{imaginária} vertical entre a margem esquerda e a margem direita, e o texto fica alinhado de forma que metade fique para um lado da linha e metade para o outro.
\end{center}

\subsection{Seção de alinhamento à direita}
\begin{flushright}
Aqui, o texto fica alinhado com a margem direita e desalinhado em relação à margem esquerda. Para alinhar um texto à \textit{direita} no \LaTeX{}, usa-se o ambiente  \textcolor{red}{flushright} que começa com o comando \textcolor{blue}{\textbackslash begin\{flushright\} } e termina com o comando  \textcolor{blue}{\textbackslash end\{flushright\}}. [1]
\end{flushright}

\subsection{Inserção}
\begin{flushleft}
Aqui, o texto fica alinhado com a margem esquerda e desalinhado em relação à margem direita. Para alinhar um texto à \textbf{esquerda} no \LaTeX{}, usa-se o ambiente  \textcolor{red}{flushleft} que começa com o comando \textcolor{blue}{\textbackslash begin\{flushleft\} } e termina com o comando  \textcolor{blue}{\textbackslash end\{flushleft\}}. [2]
\end{flushleft}

\subsubsection{Inserção de imagem}



\subsubsection{Inserção de equação}

A “Fórmula de Bhaskara” é considerada uma das mais importantes da matemática. Ela é usada para resolver as equações de segundo grau, sendo expressa da seguinte maneira: [\ref{bas:baskara}]

\begin{equation}
	 X = \frac{-b \pm \sqrt{b^2 - 4ac}}{2a}  
	\label{bas:baskara}
\end{equation}

\subsubsection{Inserção de tabela}
A trigonometria é considerada uma das áreas mais importantes da Matemática. No triângulo, os ângulos de 30º, 45º e 60º são considerados notáveis, pois estão presentes em diversos cálculos. Por isso seus valores trigonométricos correspondentes são organizados em uma tabela. 

%A identação é uma boa prática em tabelas.
\begin{table}[htb]
	\centering
	\begin{tabular}{ |c|c|c|c| } 
 		\hline
 		 x &  30$^\circ$ & 45$^\circ$ & 60$^\circ$                                    \\ \hline 
 		Seno & $\frac{1}{2}$ & $\frac{\sqrt{2}}{2}$ &  $\frac{\sqrt{3}}{2}$             \\ \hline
 		Cosseno & $\frac{\sqrt{3}}{2}$ & $\frac{\sqrt{2}}{2}$ & $\frac{1}{2}$              \\ \hline
 		Tangente & $\frac{\sqrt{3}}{3}$ & 1 & $\sqrt{3}$                                     \\ 
 		\hline
	\end{tabular}
	\caption{Tabela Seno, Cosseno e Tangente de ângulos notáveis.}
	\label{tab:Tabela}
\end{table}

\clearpage



\section{Complemento}
\subsection{Enumeração de itens em uma lista}

A enumeração de itens pode ser dada de três maneiras distintas: por pontos, por números e por descrições.

\subsubsection{Lista de pontos}
\begin{flushleft}
Utiliza-se do comando \textit{Itemize} para definir uma lista padrão cujos subitens são símbolos no formato circular •.
\vspace{0.2cm}

Assim como na lista a seguir:
\end{flushleft}

\begin{itemize}
	\item subitem
	\item subitem
	\item subitem
	\item subitem
\end{itemize}

\subsubsection{Lista enumerada}

\begin{flushleft}
Utiliza-se do comando \textit{Enumerate} para definir uma lista padrão cujos subitens são os números inteiros positivos a partir de 1.
\vspace{0.2cm}

Assim como na lista a seguir:
\end{flushleft}

\begin{enumerate}
	\item subitem
	\item subitem
	\item subitem
	\item subitem
\end{enumerate}

\begin{flushleft}

Além disso, uma sublista enumerada dentro de uma lista enumerada será organizada por ordem alfabética.

\vspace{0.2cm}

Assim como na lista a seguir:
\end{flushleft}

\begin{enumerate}
	\item subitem
	\item subitem
		\begin{enumerate}
			\item subitem
			\item subitem
			\item subitem
			\item subitem
		\end{enumerate}
	\item subitem
	\item subitem
\end{enumerate}

\subsubsection{Lista de descrições}

\begin{flushleft}
Utiliza-se do comando \textit{Description} para definir uma lista padrão cujos subitens não são identificados, possuindo apenas a identação de uma lista.
\vspace{0.2cm}

Assim como na lista a seguir:
\end{flushleft}

\begin{description}
	\item subitem
	\item subitem
	\item subitem
	\item subitem
\end{description}

\subsubsection{O pacote Enumerate}
\begin{flushleft}
O pacote enumerate, dado por \textcolor{red}{\textbackslash usepackage\{enumerate\}} possibilita a personalização da identificação dos símbolos usados para iniciar os subitens de uma lista. 

Ele adiciona os números romanos, denotados por \textbf{I}, e o grupo de letras maiúsculas (A). E também, possibilita a utilização de caracteres usuais junto com a enumeração: \textbf{I.} , \textbf{I)} , \textbf{(I)} , \textbf{I - } , ...
\end{flushleft}


\begin{enumerate}[I.]
	\item subitem
		\begin{enumerate}[A)]
			\item subitem
			\item subitem
			\item subitem
		\end{enumerate}
	\item subitem
	\item subitem
\end{enumerate}


\subsection{Tipos de letras \LaTeX:}

\begin{itemize}
	\item Normal
	\item \emph{Ênfase}
	\item \textit{Itálico}
	\item \textsl{Inclinado}
	\item \textbf{Negrito}
	\item \texttt{Máquina de Escrever}
	\item \textsc{Maiúsculas Pequenas}
	\item \textsf{Sem serifa}
	\item \underline{Sublinhado}
\end{itemize}



\subsection{Tamanhos de letras no \LaTeX:}
Os tamanhos de letras no \LaTeX podem ser variados. Assim podemos escolher tamanho de texto utilizando o comando de mesmo nome. Por exemplo, O tamanho 'footnotesize' pode ser obtido com o comando \textcolor{red}{\textbackslash \{footnotesize\}}.

\begin{itemize}
	\item {\tiny tiny}
	\item {\scriptsize scriptsize}
	\item {\footnotesize footnotesize}
	\item {\small small}
	\item {\normalsize normalsize}
	\item {\large large}
	\item {\Large Large}
	\item {\LARGE LARGE}
	\item {\huge huge}
	\item {\Huge Huge}
\end{itemize}

\end{document}
